%!TEX root = thesis.tex
%=============================================================================


\chapter{Conclusion \& Future Work}
\label{conclusion}

I began this thesis with the goal to ultimately improve \gls{hri}, by providing robot behaviors with synchronized results and thus enable them to focus on perceiving humans better.
For this I created a framework which synchronizes and combines results of components of various disciplines, such as emotion and gender recognition.
Smaller, secondary goals were also formulated:
The solution was supposed to be modular, to increase its value for research.
Naturally, the solution should also perform as good as pre-existing solutions with regards to computational speed and accuracy.
This reiterated, I consider this thesis to be a mild success.

The RoboCup@Home experiment I could show my fusion approach to not only work, but to improve the accuracy of the used \gls{ssl} component in comparison to previously employed methods.
In this particular area, this thesis can be considered fully successful.
The data set experiment however is more of a mixed bag.
It inherently showed the proposed framework to be quite modular by its ability to incorporate additional components and reuse nearly all components from the RoboCup experiment.
On the other hand, it showed the proposed framework to be considerably slower then a comparable pre-existing solution.
I briefly investigated this problem and could determine the long computation time to be not be explainable by just the components and framework alone.
As such, I suspect an undiscovered bug to be he cause of this discrepancy.
Future work may thus begin by exploring this inconsistency and -if possible- eliminate it.

%--------------------

Apart from this, the logical next step would be to use the proposed framework to fed information into a robot behavior.
This way the work I presented can be properly utilized and thus improve \gls{hri} of robot, the very motivation for this thesis.
Another path for future work could start with creating more components for the proposed framework.
A number of the components I developed (see chapter \ref{main:components:start}) can not considered to be state of the art, but rather rapidly developed proofs of concept.
For productive work and actual research in the fields of speech recognition and robotics, especially multi-modal sensor fusion, as outlined in chapter \ref{related:fusion} it may prove fruitful to incorporate better performing algorithms into the proposed framework.

Another, rather unintentional feature of the proposed framework became clear when conducting the data set experiment.
The proposed framework provides a very easy way to benchmark different components against each other by providing distinct interfaces and as such a controls the environment of components to be benchmarked very tightly.
This way algorithms to be benchmarked can be controllably fed audio and their impact upon other components inspected.
It would for example be easily conceivable to test the word error rate of different combinations of \gls{vad} and \gls{asr} components.
