%!TEX root = ./thesis.tex

\chapter{Reproducibility}

\section{Source Code}
The source code for this thesis lies in a rather vast number of git repositories.
All of these repositories are open source.
Following is a list of all repositories used in this thesis along a short description:
\captionsetup{labelformat=empty}
\begin{figure}[h]
	\begin{tabular}{ l | p{.6\textwidth} }
		Repository 				& Description	 	\\\hline
		
		MasterThesis		& Contains the latex code of this thesis		\\ 
		esiaf\_orchestrator		& Contains the source code of the Orchestrator described in chapter \ref{main:orc}		\\ 
		esiaf\_ros				& Contains the source code of the library described in chapter \ref{main:lib}	\\ 
		ma\_eval\_cost			& Contains the results of the data set experiment, the scripts used to save them and scripts to automate analysis		\\ 
		esiaf\_hyperion\_configs			& Contains configuration files for system startup for both experiments		\\ 
		esiaf\_gender\_rec		& Gender recognition component 		\\ 
		esiaf\_eval\_dummy\_nodes 				& Test nodes used in the data set experiment	 	\\ 
		AudioSegmenter					& \gls{vad} component		\\ 
		esiaf\_wav\_player	 				& Component	to feed wav files into the framework		\\ 
		rfeldhans-MA-master		& Contains general information such as User stories and raw data of RoboCup@Home experiment as well as analysis script and processed information (\texttt{eval\_raw\_data} branch)	\\ 
		esiaf\_doa				& \gls{ssl} component	\\ 
		ma\_baseline\_doa 	& \gls{ssl} component for the pre-existing pipeline used in the RoboCup@Home experiment	\\ 
		esiaf\_pocketsphinx 				& \gls{asr} component	\\ 
		esiaf\_channel\_utils				& Components for channel handling, used in the RoboCup@Home experiment	\\ 
		esiaf\_speech\_emotion\_recognition				& Emotion recognition component	\\ 
		speech-emotion-recognition				& Underlying library for the emotion and gender recognition components	\\ 
		Presentation				& Contains the presentation for this masters thesis (under the \texttt{master\_thesis} tag)	\\ 
	\end{tabular}
	\caption{}
\end{figure}\\
All these Repositories can be found under \url{https://github.com/Slothologist/MasterThesis} where MasterThesis can be any of the above repository names.

\section{Build instructions}
Due to the number of components used in this thesis and the fact that each of them is embedded in their own git repository, building this source code can be a bit challenging.
To alleviate this, I created a distribution in the CITK \cite{7759508}.
First, to bootstrap the CITK, follow their tutorial\footnote{\url{https://toolkit.cit-ec.uni-bielefeld.de/tutorials/bootstrapping}}.
Afterwards you can install the distribution named ''speechrec-pipeline'', again following their tutorial\footnote{\url{https://toolkit.cit-ec.uni-bielefeld.de/tutorials/installing}}.


Building is supported for Ubuntu 18.04 and 16.04, though under 16.04 an additional library has to be build from hand\footnote{This is documented within the README file of the relevant project under \url{https://github.com/Slothologist/esiaf_ros/}}.


\section{Start instructions}

Similar to building such a great number of components, it is equally challenging to start them.
I thus created configuration files for the Hyperion component launch engine \cite{hyperion}.
These and Hyperion itself are included in the CITK distribution.
Several Hyperion configuration files are available, including those used for both experiments.
After starting Hyperion itself, components can then be started at will through its GUI.
I recommend starting the \gls{ros} core and the Orchestrator first, then the desired intermediate components and lastly the \texttt{Audio Grabber} or \texttt{Wav Player}.