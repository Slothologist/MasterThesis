%!TEX root = ../../thesis.tex

\section{Fusion}
\label{related:fusion}
Moderns robots generally have a plethora of different sensors to perceive their environment, such as cameras, laser sensors and microphones.
Oftentimes different sensors will provide information about identical THINGS (todo: find better word), a picture provided by a camera may for example include a person, which could also be detected by a laser detection the legs of that person.
To create an optimal model of the robots world, these information must be fused.
Fusion of information can be generally split into two categories: Early fusion and late fusion.
In this section we will present both of these approaches.

\subsection{Early Fusion}
An approach can be classified as early fusion if it feeds
Early fusion is only achievable with several sensors and specifically trained or prepared classifiers.

Early fusion can provide a significant boost to classification results, as it enables the combination of raw data of several sensors with one another.
However, it can be difficult to implement correctly and 

\subsection{Late Fusion}
Late fusion encompasses several classifiers, which are typically independent from each other. 
These classifiers can take the raw data of one or several sensors and produce distinct classifications of a specific .
Afterwards, a special late fusion algorithm will take all these -potentially different- classifications and produce a definitive classification by fusion them.

Late fusion is an approach which can be used to rather easily improve an existing system, as it can build upon the already existing solutions.
However, in comparison to early fusion it lacks the ability to combine features of several sensors independently from one another.


\subsection{Fusion by Time}


- same sensor different recognizers 

- different recognizers, same recognition type

- same sensor, different recognition type, fusion by time


%-----------------------------------

- feature level fusion


- decision level fusion