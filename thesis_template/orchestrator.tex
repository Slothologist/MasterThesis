%!TEX root = thesis.tex

%=============================================================================


\chapter{Orchestrator}

\begin{itemize}
	\item manages all participating nodes
	\item determines used audio formats for transmission
	\item handles synchronization task
	\item handles meta data accumulation, introspection and acts on anomalies
\end{itemize}

meta information gathering:

\begin{itemize}
	\item how the orchestrator gathers meta information and what exactly he gathers
	\item how this information is then used
\end{itemize}

algorithms implemented for the orchestrator

\begin{itemize}
	\item minimizing amount of resampling
	\item how is synchronization/ fusion solved?
	\item feedback to user if a non-optimal configuration is found (algorithm on how to find those)
\end{itemize}

\subsection{Meta Information Gathering}
It is desirable to gather meta information about results delivered by nodes. I.e. probability of a certain result and time to compute. TTC can be used to better predict if a node will deliver a result at all. See synchronisation of results in the orchestrator algorithms chapter.

Method 1: Every results need to be enhanced with meta informations

pro

- relatively easy to implement

con

-nodes need to explicitly keep track on this information and fill them (user unfriendly)

- results bigger than they need to be

- Method 2: TTC can be computed by orchestrator, other meta information included 

pro

-TTC can be expected to always exist and be correct (can not be expected if nodes are expected to calculate it)

-node implementation is not blown up

con

-not all nodes provide results, so beamformer and filter nodes need to provide the meta information explicitly

-how would this work?

-each result message has a timestamp, in combination with the audio flow graph calculated by the orchestrator, the time from one result to the next in line is roughly the time the next node took for computation
