%!TEX root = thesis.tex
%=============================================================================
\chapter{Configurable Speech Pipeline}
\label{main:main}
In this chapter I will describe my solution to the problems presented in chapter \ref{motiv:start}.
I will first give a broad overview how and where certain tasks are handled, before describing the two major components of the pipeline, the orchestrator and library, in more detail.
%After we discussed the problems tackled by the Orchestrator and the library in more detail and thus gained deeper insight into the workings of both, we will again revert to a higher level and describe their interaction with each other.
Lastly I will present a number of components developed for my solution as a proof of concept and to introduce them, as most of them are later used in the evaluation of this thesis (see chapter \ref{eval}).

The main goal of this thesis is to provide synchronized results of audio analysis components, such as \gls{asr} and \gls{ssl} results.
To effectively provide these fused results, these results are first needed in a standardized separated form.
Additionally, to be able to synchronize separated results based on time, they need to be annotated with a timestamp.
However, this timestamp needs to fulfill a number of requirements:
as the first and most central requirement, the timestamp must not correspond to the time a given result was made available, but rather to the time when the corresponding audio signal was recorded.
This is fundamental, as each result providing component can not be expected to take the exact same time as any other.
Additionally, as the audio this timestamp corresponds to is not a singular event, but a continuous stream, it must contain a start- \& an end-time to fully and accurately describe it.
This is especially true since results may be generated on audio of vastly different length.
Consider for example a simple "yes" or "no" in contrast to a longer question, such as "Hey robot, where can I find the orange juice?".
This directly leads to another problem:
as each results now requires an annotation in form of a timestamp, each component subsequently requires this annotated audio data.

This, in turn, leads to a number of cases with a high complexity in dependencies.
Consider a setup of a \gls{ssl} component, a beamformer and an \gls{asr} component.
In this setup, the \gls{ssl} component would have to acquire audio data and annotate it with timestamps as discussed above and then provide its results.
An independent beamformer would in turn also have to acquire audio and annotate it, and then match the \gls{ssl} results provided by the respective component to its audio data.
Then it could calculate the beamformed audio signal, which would also required to be annotated with timestamps.
Lastly, the \gls{asr} component would have to collect these timestamped audio signals, produce \gls{asr} results and then annotate them with timestamps.
An additional dependency implicitly declared in this example is that the \gls{asr} component must be able to process the audio annotated with timestamps the beamformer produces.
%Most commonly used libraries for audio transmission, as we have shown in chapter \ref{related:frameworks}, do not support adding of meta-information such as timestamps, so either these components would need to be merged or must prepare a interface for communication.

Most of these requirements could, in theory, be handled by the components themselves.
However, I decided to outsource most of the work done by the components into a library, to allow them to focus on their respective goals.
This provides a number of advantages:

First, by outsourcing this work standards must inherently be defined.
As they can easily be heeded by the components, they provide benefits for the individual components and for the final synchronization of the results.
More so, by defining these standards and providing a library to heed them, the fragmentation of speech recognition components is encouraged. 
Take for example the \gls{psa} described in chapter \ref{related_work:psa}.
Not only does it contain an \gls{asr} component, but also a \gls{vad} component.
This makes sense from a developmental perspective, but impedes further development and replacing of its parts.
Even when done correctly and iteratively, after a few exchanges new technologies may need different interfaces.
By encouraging this melding of components to not occur, the resulting framework becomes highly modular and enables fast prototyping.
It also makes benchmarking individual components of such a pipeline easier, as I will later show with a number of experiments for evaluation (see chapter \ref{eval:dataset}).

Furthermore, by providing a way to easily send properly timestamped audio between components, a massive amount of work is lifted off the individual components, which in turn ensures the correct transmission of audio timestamps.
Additionally, more technical benefits include the prevention of code duplication and the reduction of bugs, which provide more time for actual research.

By embedding each component into the framework provided by the library, the proposed solution for finally synchronizing the results is also provided with standardized interfaces.
I named this solution the Orchestrator, as it not only synchronizes the results, but also keeps track of all components within the proposed pipeline.

Both library and Orchestrator work in tandem and need to communicate with each other on several occasions.
\gls{ros} was chosen as an underlying middleware, see chapter \ref{intro:ros}.
There are several reasons for this.
First, due to \gls{ros} relying on TCP, all communication can be expected to arrive, none will be lost.
Additionally, pre-existing infrastructure, such as a behavior controller used in one of the evaluations experiments already depends on \gls{ros}.
As such choosing \gls{ros} will prove beneficial for future work.

I will now continue to discuss the Orchestrator and the proposed pipelines library in more detail in designated sections.


%!TEX root = thesis.tex

%=============================================================================


\chapter{Library}

- default nodes for convenience and 
\newpage
%!TEX root = thesis.tex

%=============================================================================


\chapter{Orchestrator}

\begin{itemize}
	\item manages all participating nodes
	\item determines used audio formats for transmission
	\item handles synchronization task
	\item handles meta data accumulation, introspection and acts on anomalies
\end{itemize}

meta information gathering:

\begin{itemize}
	\item how the orchestrator gathers meta information and what exactly he gathers
	\item how this information is then used
\end{itemize}

algorithms implemented for the orchestrator

\begin{itemize}
	\item minimizing amount of resampling
	\item how is synchronization/ fusion solved?
	\item feedback to user if a non-optimal configuration is found (algorithm on how to find those)
\end{itemize}

\subsection{Meta Information Gathering}
It is desirable to gather meta information about results delivered by nodes. I.e. probability of a certain result and time to compute. TTC can be used to better predict if a node will deliver a result at all. See synchronisation of results in the orchestrator algorithms chapter.

Method 1: Every results need to be enhanced with meta informations

pro

- relatively easy to implement

con

-nodes need to explicitly keep track on this information and fill them (user unfriendly)

- results bigger than they need to be

- Method 2: TTC can be computed by orchestrator, other meta information included 

pro

-TTC can be expected to always exist and be correct (can not be expected if nodes are expected to calculate it)

-node implementation is not blown up

con

-not all nodes provide results, so beamformer and filter nodes need to provide the meta information explicitly

-how would this work?

-each result message has a timestamp, in combination with the audio flow graph calculated by the orchestrator, the time from one result to the next in line is roughly the time the next node took for computation

\newpage
%!TEX root = ../../thesis.tex

\section{Developed \& Incorporated Components}
\label{main:components:start}
In this section, components that were developed for the proposed pipeline will be discussed.
These components cover different aspects of speech recognition.

\marginnote{One good example would be a DeepSpeech component I developed\footnotemark. It makes use of a TensorFlow \cite{tensorflow} implementation of Baidus Deepspeech \cite{deepspeech}.}
\footnotetext{\url{https://github.com/Slothologist/DeepSpeech4Ros}}
Some initially planned and partially developed components were left out of this list, because it became clear they could not be included into this thesis in a meaningful way.
Thus development for these components was stopped or they were left out in favor of diverting time to more important parts of this thesis.


A handful of components are included in the library's repository, as they provide utility functions or are commonly used.
None of these components provide any information to the \textit{Orchestrator}, apart from registering.
These components are:
\begin{itemize}[leftmargin=1in]
	\item[\textit{Audio Grabber}] The \textit{Audio Grabber} is the most fundamental node, as it is the basis of most pipeline configurations.
	It is able to grab audio from a microphone via \gls{alsa} and feed it into the pipeline.

	\item[\textit{Audio Player}] This component is the \textit{Audio Grabbers} counterpart, as it receives audio data from the pipeline and outputs it via \gls{alsa} through a speaker.
	As such, it is mostly used for debugging purposes and to enable quick auditory microphone checks.

	\item[\textit{Channel Splitter}] As discussed in section \ref{main:lib:formats}, the proposed pipeline's library does not support mixing of channels.
	The \textit{Channel Splitter} is used to mitigate this absence by receiving a multichannel audio signal and produces a corresponding number of single channel outputs.
	One possible use case is to split a multichannel audio needed for \gls{ssl} into single channel audio usable for speech recognition, when no beamforming is desired.

	\item[\textit{Recorder}] The \textit{Recorder} is able to write audio it receives via the proposed pipeline to a file.
	Its foremost usage is to save audio it receives via the proposed pipeline to a file for later inspection or analysis.
	This can either be done for documentation purposes or, as was mostly done during development, to check if the pipeline transmitted and resampled audio correctly.
\end{itemize}
Additional components not included in the libraries repository cover the topics of:

\subsubsection{Wav Player}
\label{main:components:wav}
The \textit{Wav Player}\footnote{\url{https://github.com/Slothologist/esiaf_wav_player}} fulfills the same role as the \textit{Audio Grabber}, as it feeds audio into the pipeline.
However, it will read a given set of wav files in and output them into the pipeline.
It is capable of either producing silence in between each wav file or waiting for a time before playing the next file.
As such its predominant use case is to enable evaluations of other components with the help of data sets.

\subsubsection{Sound Source Localization}
\label{main:components:ssl}
I developed a component\footnote{\url{https://github.com/Slothologist/esiaf_doa}} which performs \gls{ssl}.
It uses the python library \texttt{pyroomacoustics} \cite{pyroomacoustics}, and specifically its SRP algorithm, though usage of all other \gls{ssl} algorithms provided by \texttt{pyroomacoustics} can be configured.
For each sound chunk the component receives via the proposed framework, it creates \gls{ssl} results and then sends them to the Orchestrator.

A slight variation of this node\footnote{\url{https://github.com/Slothologist/ma_baseline_doa}}  will be used in one of the experiments of the evaluation chapter (see \ref{eval:task_start}).
This variant will, instead of publishing results for each sound chunk it receives, store these results in a queue.
To communicate the results it provides a \gls{ros} service which allows any component to ask for results which occurred in a specified time frame.
Additionally, this variant will receive audio not from the proposed framework, but instead capture it from a microphone via \gls{alsa}.

\subsubsection{Voice Activity Detection}
\label{main:components:vad}
The \gls{vad} I developed\footnote{\url{https://github.com/Slothologist/AudioSegmenter}} is a reimplementation of a pre-existing algorithm present in the \gls{psa} (see chapter \ref{related_work:psa}).
Its segmentation is therefore virtually identical to the \gls{psa}'s.

If it detects the end of a segmentation, it will enhance the corresponding audio chunk with a \texttt{segmentation\_ended annotation}, as described in chapter \ref{main:lib:augmented_audio_msg}, before sending it.
Audio will not be transmitted to the next node(s) if it were not found to include speech.

\subsubsection{Automatic Speech Recognition}
\label{main:components:ps}
The \gls{asr} component I developed\footnote{\url{https://github.com/Slothologist/esiaf_pocketsphinx}} is a very simple wrapper around the python wrapper of \gls{ps}.
It will feed all audio it receives into \gls{ps}, which will then dynamically produce results.
These results are however only used if a \gls{vad} signals the end of a segment.
As such, this \gls{ps} component requires a \gls{vad} to work.
Whenever a result is produced, it is sent to the Orchestrator via published \gls{ros} message (see chapter \ref{main:lib:message_types}).

\subsubsection{Emotion Recognition}
\label{main:components:emotion}
The emotion recognition component I developed\footnote{\url{https://github.com/Slothologist/esiaf_speech_emotion_recognition}} makes use of \texttt{Speech Emotion Recognition} \cite{speech-em-rec}.
\texttt{Speech Emotion Recognition} uses a neural net to extract the emotional state of a person based on their speech.
As such, it is capable to categorize speech into three emotions: angry, happy, sad.
Additionally, speech can be categorized as neutral.
Small changes were required for it to meet all the proposed framework's expectations, so I improved it to include a probability for each result. %TODO
Each sound chunk this component acquires will be fed into \texttt{Speech Emotion Recognition} which will return an emotion and probability for each of these sound chunks.
These results are then sent to the Orchestrator for further synchronization.

\subsubsection{Gender Recognition}
\label{main:components:gender}
The gender recognition component\footnote{\url{https://github.com/Slothologist/esiaf_gender_rec}} I created is actually a somewhat new development.
It is based upon \texttt{Speech Emotion Recognition}, as it uses a nearly identical neural net and dataset.
However, I adjusted the output dimension of the neural net and reorganized its training data to classify either to male or to female, instead of the four emotions.
It also includes my changes, so its results include a probability.
I retrained the net and achieved a training accuracy of 99.63\% while maintaining a test accuracy on unseen data of 91.20\%.
Similar to the described emotion recognition component, it will produce a gender result for each sound chunk it receives along with a probability for it and send them to the Orchestrator.

It should be noted that the data set was quite small with only 339 audio samples, 188 of which by 5 female speakers , 151 from 5 male.
I suspect this being the cause for it to not achieve the level of accuracy on real world data that it achieved on the test data.
This observation is however anecdotally and was not evaluated formally.
However, I was satisfied with this solution, as it produces somewhat reasonable results.

This is especially true given the fact that it is not a part of this thesis to develop new approaches to recognition of gender, emotion or speech.
I chose this course of action with this specific component however, because an almost feasible component in the form of the described emotion recognizer was already in place and retraining of its neural net seemed to be a quick and easy way to produce a new component.
The alternative would have been to restructure and adjust a different approach, which may not have worked at all.







