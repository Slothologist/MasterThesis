%!TEX root = ../thesis.tex

\section{RoboCup@Home}
The RoboCup % generelle beschriebung, kurz

RoboCup@Home % genauere beschreibung von @home

- how is this relevant?

One of the tasks tested in RoboCup@Home of 2017 and 2018, the ''Speech and Person Recognition'' task \cite{rulebook_2018}, is of particular interest for this work, as it mainly tests the basic speech recognition ability of the robots. 
As it was used in several RoboCup events over two years, and on dozens of robots, it is an almost ideal test to evaluate any speech recognition system on a robot.
Consequently, we will use it to evaluate the performance of the proposed pipeline later in chapter %todo 

\subsection{RoboFEI}

The RoboFEI Team from Sao Paulo uses a specific microphone configuration to handle problems caused by highly mobile sound source localization microphones. %todo: cite
Their robots head is attached to a rotatable pipe, which enables it to spin around its axis, while the robots microphone array is attached to a static shaft running through that pipe which does not allow it to rotate.

As such, the position and movement of the robots head are not relevant for sound source localization results and beamforming.
Nevertheless, the robots own position and movement needs to be taken into account when performing 


\subsection{Walking Machine}
no pipeline for that matter, but an all-inclusive solution, ODAS
%todo cite

\subsection{Tech United}
benutzen anscheinend beamforming, reimplementieren das für den HSR
%todo: cite

\begin{itemize}
	\item 
\end{itemize}
