%!TEX root = thesis.tex

\chapter{Related Work}

In this chapter a number of related (research-) projects will be presented and discussed. 


%-------------------------------------------------------------------------------------
\section{JACK Audio}

The Jack audio connection kit \cite{JACK} is a protocol for audio transmission between components with the explicit requirement for a low latency, to enable real time capture, playback and editing of sound. 
It provides an API to transmit audio between programs. 

Several implementations of JACK servers exist, notably JACK1 and JACK2. 
Differences between implementations are quite small and boil down to a handful of optional features and support for some operating systems which will most likely not matter for most users. % see https://github.com/jackaudio/jackaudio.github.com/wiki/Q_difference_jack1_jack2
Since all implementations provide the same interfaces, components are indifferent to the implementation of the JACK server used. 

JACK requires components to use a specific sample rate, sample size and audio format, which can however be adjusted before starting the JACK server. 
As such, any component which requires a different audio format has to handle resampling and converting on its own. 

JACK does not allow for additional information to transmitted with the raw audio it transports, so any additionally information must be transmitted and synchronized separately with another middleware. 

JACK is able to transmit audio over network, but generally requires all participating computers to run an instance of the same server implementation. 
An additional piece of software is needed to then create the connection between the different server instances. 
This complexity results in a moderate amount of setup and maintenance time required to handle an ever changing setup of computers and components a robot, especially in a research context, requires.

JACK requires all components which use its API to process audio in a specific and rather short time. 
Audio is written to and read from JACK via buffers, which will be in turn read and written by JACK in specific intervals.
If a component is not able to read or produce audio fast enough, JACK will overwrite or read old audio respectively, so audio is lost. 
This is important for applications which result in humans hearing the transmitted audio directly, to reduce latency and negate long or at least noticeable periods of silence.

\subsection{Problems regarding distributed Speech recognition applications}
For speech recognition purposes however, this audio frame dropping produces a series of problems. 
For once, transitions between audio frames become non-steady, which results in artifacts when frequency analysis or other feature extraction is employed.
Feature extraction algorithms using deep learning for example may not have been trained with audio which misses frames, and produce incomprehensible or unexpected results.

Furthermore, dropping audio frames poses the more fundamental problem of loosing information, which complicates recognition and worses results. 
To counter this, developers using JACK must take special care of reading and writing audio, which creates an additional requirement.
This leads to JACK being rather unsuitable as a middleware for a number of speech recognition applications, especially in a research context, where components are often exchanged.

\subsection{Gstreamer \& ALSA/ PulseAudio}

Here we will briefly discuss three other audio frameworks, namely ALSA, Pulseaudio and Gstreamer.

The Advanced Linux Sound Architecture (ALSA) 
It provides drivers and access to hardware sound sources and sinks, typically -but not limited to- microphones and speakers.

PulseAudio builds on top of ALSA and provides an abstraction layer as well as additional interfaces and features, such as seamless resampling and audio transmission over network.

Gstreamer \cite{Gstreamer}

\begin{itemize}
	\item de facto standard on every linux machine, but available on several other operating systems
	\item provides access to hardware sound sources and sinks
\end{itemize}

Most of the presented frameworks can be used alongside each other, as they oftentimes have interfaces for each other. 
A combination of the presented frameworks may come to mind but most of the discussed problems will just add up instead of cancel each other out.

%-------------------------------------------------------------------------------------


\section{RoboCup@Home}
The RoboCup % generelle beschriebung, kurz

RoboCup@Home % genauere beschreibung von @home

- how is this relevant?

One of the tasks tested in RoboCup@Home of 2017 and 2018, the ''Speech and Person Recognition'' task \cite{rulebook_2018}, is of particular interest for this work, as it mainly tests the basic speech recognition ability of the robots. 
As it was used in several RoboCup events over two years, and on dozens of robots, it is an almost ideal test to evaluate any speech recognition system on a robot.
Consequently, we will use it to evaluate the performance of the proposed pipeline later in chapter %todo 

\subsection{RoboFEI}

The RoboFEI Team from Sao Paulo uses a specific microphone configuration to handle problems caused by highly mobile sound source localization microphones. %todo: cite
Their robots head is attached to a rotatable pipe, which enables it to spin around its axis, while the robots microphone array is attached to a static shaft running through that pipe which does not allow it to rotate.

As such, the position and movement of the robots head are not relevant for sound source localization results and beamforming.
Nevertheless, the robots own position and movement needs to be taken into account when performing 


\subsection{Walking Machine}
no pipeline for that matter, but an all-inclusive solution, ODAS
%todo cite

\subsection{Tech United}
benutzen anscheinend beamforming, reimplementieren das für den HSR
%todo: cite

\begin{itemize}
	\item 
\end{itemize}
