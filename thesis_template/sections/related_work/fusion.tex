%!TEX root = ../../thesis.tex

\section{Fusion}
\label{related:fusion}
Moderns robots generally have a plethora of different sensors to perceive their environment, such as cameras, laser sensors and microphones.
Oftentimes different sensors will provide information about identical characteristics, a picture provided by a camera may for example include a person, which could also be detected by a laser detection the legs of that person.
To create an optimal model of the robots world, these information must be fused.
In this section we will present a number of approaches for this fusion.
Most of these approaches describe multisensor systems.
Though our interest lies in the field of speech recognition, which typically makes use of just one sensor -the microphone-, these can easily be applied by broadening the definition of sensor from physical ones to also include components as a sort of software based sensors.

\cite{TURK2014189} and \cite{PORIA201650} divide between feature and decision level fusion within multimodal systems.
As such they only cover classification of a single characteristic with data of several sensors.
Feature and decision level fusion differ in the level on which the fusion happens.
Late fusion encompasses several unique classifiers, which are typically independent from each other. 
These classifiers can take the raw data of one or several sensors and produce distinct classifications of a shared characteristic.
Afterwards, a special late fusion algorithm will take all these -potentially different- classifications and produce a definitive classification.
In contrast to that, early fusion makes use of only a single classifier, which is fed with raw data of several sensors and produce a classification directly from that.
Early fusion is thus only achievable with several sensors and specifically trained or prepared classifiers.

\cite{287895} provide an algorithm to fuse the asynchronous results of multiple sensors which focus on th same characteristic.
They thus try to extract the most likely states of the observed system.
\cite{10.1117/12.138164} solves the issue of synchronizing results by batching all data within a specific time interval and then compressing them. 
In contrast to that, \cite{4383603} uses a recursive algorithm in conjunction with a Kalman filter to fuse asynchronous multisensor data and thus better estimate the state of a given system.
In \cite{1468761} asynchronous sensor data are fused without the need synchronization.
This however necessitates all sensors to perceive the same characteristic.