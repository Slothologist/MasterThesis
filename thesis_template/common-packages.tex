\usepackage{csquotes} % Context sensitive quotation facilities. Recommended by babel and should be loaded before babel.

\usepackage[english]{babel} % Sets the language used. Essential for proper hyphenation. Translates key words like 'figure' or 'table'. Note that 'english' is American English.

\usepackage{latexsym,amsmath,amssymb,amsthm,amscd} % Provides math enviroments, math symbols and more things useful for math.

\usepackage{mathtools} % Enhances amsmath and provides further mathematical tools.

\usepackage{graphicx} % Provides inclusion of graphics and a proper interface for '\in­clude­graph­ics'.
% Su­per­sedes 'epsfig' and 'graphics'.

\usepackage{enumitem} % Provides control over list environments. Supersedes 'enumerate', which gives enumerate environment an optional argument which determines the style in which the counter is printed.

%\let\newfloat\undefined % Hack to make floatrow work with memoir.
%\usepackage{floatrow} % Handles alignment of floats (figures), with centering as default.

\usepackage{xspace} % Ugly hack that sometimes helps with otherwise missing spaces.

\usepackage[backgroundcolor=white,linecolor=smartblue,bordercolor=smartblue,textsize=footnotesize]{todonotes} % Todo notes.

% Use sans-serif font in todos to clearly distinguish it from other text
\makeatletter
\renewcommand{\todo}[2][]{\@bsphack\@todo[#1]{\sffamily{#2}}\@esphack\ignorespaces}
\makeatother

\usepackage[style=alphabetic,backend=biber]{biblatex} % Bibliography.

%\usepackage{algorithm} % Algorithms.
% Use default Memoir floats for algorithms
\newcommand{\algorithmname}{Algorithm}
\newcommand{\listalgorithmname}{List of Algorithms}
\newlistof{listofalgorithms}{loa}{\listalgorithmname}
\newfloat[chapter]{algorithm}{loa}{\algorithmname}
\newfixedcaption{\falgcaption}{algorithm}
\newlistentry[chapter]{algorithm}{loa}{0}
\cftsetindents{algorithm}{0em}{2.3em}
\makeatletter
\g@addto@macro\insertchapterspace{\addtocontents{loa}{\protect\addvspace{10pt}}}
\makeatother

%\usepackage{algpseudocode} % Algorithms.

\usepackage{varioref} % Automatically locates references on other pages. Load before hyperref.

%\usepackage[hidelinks]{hyperref} % Clickable references.
\usepackage[pdfencoding=auto,pdfborderstyle={/S/U/W 1},linkbordercolor=ref,colorlinks,linkcolor=ref,citecolor=cite,urlcolor=url]{hyperref} % Clickable references.

\usepackage[all]{hypcap} % Anchors links to the beginning of their respective floats. Load after hyperref.

\usepackage[noabbrev,capitalize,nameinlink]{cleveref} % Names references automatically. Load after hyperref.

\usepackage[toc,acronym]{glossaries} % Glossary.

\usepackage{booktabs}
\usepackage{multirow}

\usepackage{pgfplots}
\pgfplotsset{compat=1.8}
\pgfplotscreateplotcyclelist{smart}{%
  tangoplum,semithick,every mark/.append style={fill=tangoplum!80!black},mark=*\\%
  tangogreen,semithick,every mark/.append style={fill=tangogreen!80!black},mark=square*\\%
  tangoorange,semithick,every mark/.append style={fill=tangoorange!80!black},mark=otimes*\\%
  tangoblue,semithick,mark=star\\%
  tangobutter,semithick,every mark/.append style={fill=tangobutter!80!black},mark=diamond*\\%
  tangored,semithick,every mark/.append style={solid,fill=tangored!80!black},mark=*\\%
}
\pgfplotsset{cycle list name=smart}

\usepackage{listings}

\definecolor{mygreen}{rgb}{0,0.6,0}
\definecolor{mygray}{rgb}{0.5,0.5,0.5}
\definecolor{mymauve}{rgb}{0.58,0,0.82}

\lstset{ %
  backgroundcolor=\color{white},   % choose the background color; you must add \usepackage{color} or \usepackage{xcolor}
  basicstyle=\footnotesize,        % the size of the fonts that are used for the code
  breakatwhitespace=false,         % sets if automatic breaks should only happen at whitespace
  breaklines=true,                 % sets automatic line breaking
  captionpos=b,                    % sets the caption-position to bottom
  commentstyle=\color{mygreen},    % comment style
  deletekeywords={...},            % if you want to delete keywords from the given language
  escapeinside={\%*}{*)},          % if you want to add LaTeX within your code
  extendedchars=true,              % lets you use non-ASCII characters; for 8-bits encodings only, does not work with UTF-8
  frame=single,	                   % adds a frame around the code
  keepspaces=true,                 % keeps spaces in text, useful for keeping indentation of code (possibly needs columns=flexible)
  keywordstyle=\color{blue},       % keyword style
  language=C++,                	% the language of the code
  otherkeywords={*,...},           % if you want to add more keywords to the set
  numbers=left,                    % where to put the line-numbers; possible values are (none, left, right)
  numbersep=5pt,                   % how far the line-numbers are from the code
  numberstyle=\tiny\color{mygray}, % the style that is used for the line-numbers
  rulecolor=\color{black},         % if not set, the frame-color may be changed on line-breaks within not-black text (e.g. comments (green here))
  showspaces=false,                % show spaces everywhere adding particular underscores; it overrides 'showstringspaces'
  showstringspaces=false,          % underline spaces within strings only
  showtabs=false,                  % show tabs within strings adding particular underscores
  stepnumber=1,                    % the step between two line-numbers. If it's 1, each line will be numbered
  stringstyle=\color{mymauve},     % string literal style
  tabsize=2,	                   % sets default tabsize to 2 spaces
  title=\lstname,                   % show the filename of files included with \lstinputlisting; also try caption instead of title
  literate=%
    {Ö}{{\"O}}1
    {Ä}{{\"A}}1
    {Ü}{{\"U}}1
    {ß}{{\ss}}1
    {ü}{{\"u}}1
    {ä}{{\"a}}1
    {ö}{{\"o}}1
    {~}{{\textasciitilde}}1
}

\renewcommand{\lstlistingname}{Quellcode}
